\documentclass{homework}
\course{Math 6701}
\author{Jim Fowler}
\title{Assigned Readings}
\usepackage{hyperref}


\begin{document}
\maketitle

\begin{inspiration} If I only knew how to get the mathematicians
interested in transformation groups and their applications to
differential equations. I am certain, absolutely certain in my case,
that these theories in the future will be recognized as fundamental.
\byline{Sophus Lie, in a letter to Adolph Mayer}
\end{inspiration}

\section{Terminology}

\begin{problem} The word ``distribution'' is used throughout
mathematics.  What is meant by a \textbf{distribution} in this course?
\end{problem}

\begin{problem} What does it mean to say that a distribution is
\textbf{completely integrable}?
\end{problem}

\begin{problem} What is a \textbf{Lie subgroup}?
\end{problem}

\section{Numericals}

\begin{problem} Consider the form $\omega = dz - y \, dx \in
\Omega^1(\R^3)$.  This is a \textbf{contact form}.  Compute $\omega
\wedge d\omega$.

  Is the distribution $\mathcal{D} = \ker \omega$ completely
integrable?
\end{problem}

\section{Exploration}

\begin{problem} Again consider the form $\omega = dz - y \, dx \in
\Omega^1(\R^3)$. A curve $\gamma : [a,b] \to \R^3$ is a
\textbf{Legendrian curve} if $\gamma'(t) \in \ker \omega$ whenever $t
\in (a,b)$. What points in $\R^3$ can be connected by a sequence of
Legendrian curves?
\end{problem}

\begin{problem} On $M = \R^2 \setminus \{ (0,0) \}$, consider the
vector fields $A, B \in \Gamma(TM)$ given by
  \begin{align*}
    A_{(x,y)} &= \frac{1}{\sqrt{x^2 + y^2}} \left( \phantom{-} x \, \frac{\partial}{\partial x} + y\, \frac{\partial}{\partial y} \right), \\
    B_{(x,y)} &=  \frac{1}{\sqrt{x^2 + y^2}} \left( -y \,\frac{\partial}{\partial x} + x\, \frac{\partial}{\partial y} \right). \\
  \end{align*} Around some point $p \in M$, is there a neighborhood $U
\ni p$ and coordinates $a, b : U \to \R$ so that $A =
\frac{\partial}{\partial a}$ and $B = \frac{\partial}{\partial b}$?
Why or why not?
\end{problem}

\begin{problem} Suppose $G$ is a Lie group with Lie algebra
$\mathfrak{g} = T_e G$, and suppose there is a Lie subalgebra
$\mathfrak{h} \subset \mathfrak{g}$.  Build a completely integrable
distribution using $\mathfrak{h}$.
Can you use this distribution to find a Lie subgroup $H$ of $G$ with
$\mathfrak{h} = T_e H$?
\end{problem}

\section{Prove or Disprove and Salvage if Possible (PODASIP)}

\begin{problem} If $X, Y \in \Gamma(TG)$ are left-invariant vector
fields on the Lie group $G$, then their bracket $[X,Y]$ is also
left-invariant.
\end{problem}

\begin{problem}\label{cartan-formula}For a vector field $X \in \Gamma(TM)$ and $k$-form $\omega \in \Omega^\star(M)$,
\[
  {\mathcal {L}}_{X}\omega =d(\iota _{X}\omega )+\iota _{X}d\omega.
\]
This is \textbf{Cartan's magic formula}.
\end{problem}

\end{document}
