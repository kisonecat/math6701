\documentclass{homework}
\course{Math 6701}
\author{Jim Fowler}
\title{Assigned Readings}
\usepackage{hyperref}


\DeclareMathOperator{\GL}{GL}

\begin{document}
\maketitle

\begin{inspiration}
A connection on a bundle\ldots is a rule that identifies fibers of the bundle along paths in the base space.
  \byline{Urs Schreiber in the \textit{nLab}}
\end{inspiration}

\section{Terminology}

\begin{problem} For a smooth manifold $M$, what is meant by a
\textbf{connection} on (or is it ``in'') the vector bundle $E \to M$?
\end{problem}

\begin{problem} Define $\Omega^k(M;E)$, the $E$-valued $k$-forms.
\end{problem}

\begin{problem} For a connection $\nabla$ on the vector bundle $E \to
M$, define the curvature $F_\nabla$ as a 2-form with values in $\End(E)$.
\end{problem}

\section{Numericals}

\begin{problem}
  On a smooth manifold $M$ with an $n$-dimensional vector bundle $\pi : E \to M$, suppose $E$ is trivial over open subsets $U, V \subset M$, with trivializations
  \begin{align*}
    \varphi_U &: \pi^{-1}(U) \to U \times \R^n, \\
    \varphi_V &: \pi^{-1}(V) \to V \times \R^n,
  \end{align*}
  and transition function $g_{UV} : U \cap V \to \GL_n(\R)$.

  When there is a connection on the bundle $E$, we have connection
forms $\omega_U$ and $\omega_V$ which are $n \times n$ matrices of
$1$-forms on $U$ and $V$, respectively.

  Verify that \(
    \omega_V = {g_{UV}}^{-1} \cdot \omega_U \cdot g_{UV} + {g_{UV}}^{-1} \cdot dg_{UV}
  \).
\end{problem}

\section{Exploration}

\begin{problem} In light of the previous calculation, it is said that
the connection form ``does not transform tensorially.''  What does
that mean?
\end{problem}

\begin{problem} Verify that the curvature form \textit{does} transform
as a tensor.
\end{problem}

\begin{problem}
  Regard a connection $\nabla$ on the bundle $E \to M$ as a map $\nabla : \Omega^0(M;E) \to \Omega^1(M;E)$.  Then we can extend this to an \textbf{exterior covariant derivative} 
  \[
    d_{\nabla }:\Omega ^{k}(M;E)\to \Omega ^{k+1}(M;E)
  \]
  via the rule
  \[
    d_{\nabla }(\omega \otimes s)=d\omega \otimes s + (-1)^{\deg \omega} \omega \wedge \nabla s.
  \]
  Recall that $d^2 = 0$, and consequently the $\Omega^\star(M)$ forms a chain complex.

  Is the same true for $\Omega^\star(M;E)$ with $d_\nabla$?  What is $d_\nabla \circ d_\nabla$?
\end{problem}

\section{Prove or Disprove and Salvage if Possible (PODASIP)}

\begin{problem}
 The curvature two-form $F_\nabla$ is $d_\nabla$-closed, i.e., \(
    d_\nabla F_\nabla = 0
   \).
\end{problem}

\end{document}
