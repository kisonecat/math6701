\documentclass{homework}
\course{Math 6701}
\author{Jim Fowler}
\title{Assigned Readings}
\usepackage{hyperref}


\DeclareMathOperator{\GL}{GL}
\DeclareMathOperator{\SU}{SU}
\DeclareMathOperator{\U}{U}

\begin{document}
\maketitle

\begin{inspiration} The salient fact about the Levi-Civita parallelism
is the result that it is the parallelism, and not the Riemannian
metric, which accounts for most of the properties concerning
curvature.  \byline{Shiing-Shen Chern}
% in notes for math 352a at uchicago, spring 1952,
\end{inspiration}

\section{Terminology}

\begin{problem}
  For a $2n$-by-$2n$ skew-symmetric matrix $M$, what is the \textbf{Pfaffian} of $M$?
\end{problem}

\begin{problem}
  For a complex vector bundle $E \to M$, define the \textbf{Chern class}
  \[
    c_k(E) \in H^{2k}(M;\R).
  \]
\end{problem}

\begin{problem}
  For a real oriented $\R^{2n}$-bundle $E \to M$, define the \textbf{Euler class}
  \[
    e(E) \in H^{2n}(M;\R).
    \]
\end{problem}

\section{Numericals}

\begin{problem}
  For a connection $\nabla$ with curvature two-form $F_\nabla$, compute
    $d(\trace F_\nabla)$.
\end{problem}

\section{Exploration}

\begin{problem} Suppose the transition functions for a certain vector
bundle $E \to M$ are in a Lie group $G$, i.e., for an open cover $\{
U_i \}$ of $M$ trivializing the bundle, we have transition functions
$g_{ij} : U_i \cap U_j \to G$.

Suppose a connection $1$-form $\omega$ on $E$ takes values in the
Lie algebra $T_e G$.  Does this transform in a reasonable way?
\end{problem}

\begin{problem}
  For $M \in T_{\id} \SO(n)$, what is the trace of $M$?

  What does this imply about the characteristic class $\trace F_\nabla$?
\end{problem}

\begin{problem}
 For a complex bundle $E \to M$, can the transition functions be chosen to lie in the group $\U(n)$?  
\end{problem}

\begin{problem}
  For $M \in T_{\id} \U(n)$, what can be said about the trace of $M$?

  What does this imply about the characteristic class $\trace F_\nabla$?
\end{problem}

\section{Prove or Disprove and Salvage if Possible (PODASIP)}

\begin{problem}
  For complex bundles $E \to M$ and $F \to M$, 
    $c_k(E \oplus F) = c_k(E) c_k(F)$.
\end{problem}

\begin{problem}
  Suppose $E \to M$ is an oriented $\R^{2n}$-bundle with $e(E) \neq 0$ and $f \in \Gamma(E)$.

  Then there exists $p \in M$ so that $f(p) = 0$.
\end{problem}

\end{document}
