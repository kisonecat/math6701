\documentclass{homework}
\title{Assigned Readings}
\usepackage{hyperref}

\author{Jim Fowler}
\course{Math 6701}
\date{Week 5: Forms on manifolds}

\usepackage{draftwatermark}
\SetWatermarkText{Draft}
\SetWatermarkScale{5}

\begin{document}
\maketitle

Last week, we reviewed some ``exterior algebra'' and made an important
observation that the exterior algebra is a quotient of tensors on a
vector space, while we also want to regard differential forms as
acting on vector fields.  Technically, last week we were thinking
about differential forms on $\mathbb{R}^n$, so this week we will have
to discuss further what we mean by forms on a smooth manifold.  To dig
deeper, open up Morita's \textit{Geometry of Differential Forms} and
read
\begin{itemize}
\item 2.2 Various operations on differential forms
\end{itemize} The ``various operations'' include wedge products, the
exterior derivative, and even the ``Lie derivative.''  A puzzle last
week was that we can differentiate a function at a point but somehow
we had trouble differentiating a vector field at a single point.  And
yet, we \textit{can} differentiate a vector field along another vector
field --- recall was one perspective on the Lie bracket.  This week we
finally meet the Lie derivative $\mathcal{L}_X$ which differentiates a
suitable geometric object (like other tensors) along the flow given by
a vector field $X$.  And that means we'll have to say more about
``flow'' as well.  Next week is the Frobenius theorem which will built
on our discussion of vector fields and forms.

In any graduate course, a challenge can be the pace and the height of
the terminology --- each week builds on our previous learning, so if
you are feeling even a bit lost or uncertain, please let me know right
away.p

\end{document}
