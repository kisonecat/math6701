\documentclass{homework}
\title{Assigned Readings}
\usepackage{hyperref}

\author{Jim Fowler}
\course{Math 6701}
\date{Week 4: Exterior algebra}

\begin{document}
\maketitle

Given that the title of our textbook is \textit{Geometry of
Differential Forms}, the topic of differential forms must be central
to our course.  Given the importance of this topic, we are scheduled
to spend this week on ``exterior algebra'' and exterior
differentiation, thereby setting up the foundations for a deeper study
of differential forms next week.  This slower schedule also provides a
time for us to consolidate our prior learning and review topics that
may have been confusing on the first pass.

Thus, from Morita's \textit{Geometry of Differential Forms}, read
\begin{itemize}
\item 1.5 Fundamental facts concerning manifolds
\item 2.1 Definition of differential forms
\end{itemize}
Working through differential forms means understanding some
\textbf{multilinear algebra}, so Problem Set 3 included a question
about the Veronese embedding in terms of symmetric tensors.

Note that throughout this course, there are different sorts of
``derivative'' in play.  Last week, we saw the bracket of vector
fields.  This week, we will see the exterior derivative.  We can take
partial derivatives of functions\ldots what is stopping us from taking
the partial derivative of a vector field at a point?  How does
exterior differentiation get around obstacle?

\end{document}
