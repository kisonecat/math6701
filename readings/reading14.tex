\documentclass{homework}
\title{Assigned Readings}
\usepackage{hyperref}

\author{Jim Fowler}
\course{Math 6701}
\date{Week 14: Applications}

\begin{document}
\maketitle

This is a shortened week---indeed, it is just one day---because of
Thanksgiving break.  Read
\begin{itemize}
\item 5.7 Applications of characteristic classes
\end{itemize}
from Morita's \textit{Geometry of Differential Forms}.

With all the effort put into building these characteristic classes,
there's a natural question: what's it all good for?  Superficially,
\textit{any} nontrivial characteristic class affirms the claim that
the bundle is not the trivial bundle\ldots but beyond that, exactly
what sort of twisting are these classes measuring?

One answer arrives with a deeper study of the Euler class which, when
integrated over the manifold, yields the Euler characteristic.  This
suggests a general mechanism for converting (top dimensional!)
cohomology classes to numbers---just integrate!  For Pontrjagin
classes, we get the Pontrjagin numbers, which ultimately connects to
cobordism, answering the question of when an $n$-manifold bounds an
$(n+1)$-manifold.


\end{document}
